\documentclass[a4paper]{article} 
\addtolength{\hoffset}{-2.25cm}
\addtolength{\textwidth}{4.5cm}
\addtolength{\voffset}{-3.25cm}
\addtolength{\textheight}{5cm}
\setlength{\parskip}{0pt}
\setlength{\parindent}{0in}

%----------------------------------------------------------------------------------------
%	PACKAGES AND OTHER DOCUMENT CONFIGURATIONS
%----------------------------------------------------------------------------------------
\usepackage{bm} % Bold math symbols, !must be loaded before unicode-math!
\usepackage{blindtext} % Package to generate dummy text
\usepackage{enumerate} % Enumerate with redefinable labels
\usepackage{enumitem} % Enumerate with redefinable labels
\renewcommand{\thesubsection}{\thesection.\alph{subsection}} % Enumerate subsections with letters
\renewcommand{\thesubsubsection}{\thesubsection.\roman{subsubsection}} % Enumerate subsubsections with roman numerals
\usepackage{charter} % Use the Charter font
\usepackage[utf8]{inputenc} % Use UTF-8 encoding
\usepackage{microtype} % Slightly tweak font spacing for aesthetics
\usepackage[english, ngerman]{babel} % Language hyphenation and typographical rules
\usepackage{amsthm, amsmath, amssymb} % Mathematical typesetting
\newcommand{\argmax}[1]{\underset{#1}{\text{arg max}}\;}
\newcommand{\argmin}[1]{\underset{#1}{\text{arg min}}\;}
\newcommand{\mb}[1]{\mathbf{#1}}  % Bold math symbols shorthand
\usepackage{float} % Improved interface for floating objects
\usepackage[final, colorlinks = true, 
            linkcolor = black, 
            citecolor = black]{hyperref} % For hyperlinks in the PDF
\usepackage{graphicx, multicol} % Enhanced support for graphics
\usepackage{xcolor} % Driver-independent color extensions
\usepackage{marvosym, wasysym} % More symbols
\usepackage{rotating} % Rotation tools
\usepackage{censor} % Facilities for controlling restricted text
\usepackage{listings, style/lstlisting} % Environment for non-formatted code, !uses style file!
\usepackage{pseudocode} % Environment for specifying algorithms in a natural way
\usepackage{style/avm} % Environment for f-structures, !uses style file!
\usepackage{booktabs} % Enhances quality of tables
\usepackage{tikz-qtree} % Easy tree drawing tool
\usepackage{todonotes} % Tool to insert TODOs
\tikzset{every tree node/.style={align=center,anchor=north},
         level distance=2cm} % Configuration for q-trees
\usepackage{style/btree} % Configuration for b-trees and b+-trees, !uses style file!
\usepackage[backend=biber,style=numeric,
            sorting=nyt]{biblatex} % Complete reimplementation of bibliographic facilities
% \addbibresource{ecl.bib}
\usepackage{csquotes} % Context sensitive quotation facilities
\usepackage[yyyymmdd]{datetime} % Uses YEAR-MONTH-DAY format for dates
\renewcommand{\dateseparator}{-} % Sets dateseparator to '-'
\usepackage{fancyhdr} % Headers and footers
\pagestyle{fancy} % All pages have headers and footers
\fancyhead{}\renewcommand{\headrulewidth}{0pt} % Blank out the default header
\fancyfoot[L]{} % Custom footer text
\fancyfoot[C]{} % Custom footer text
\fancyfoot[R]{\thepage} % Custom footer text
\newcommand{\note}[1]{\marginpar{\scriptsize \textcolor{red}{#1}}} % Enables comments in red on margin
\newcommand{\iverson}[1]{\ensuremath{[#1]}} % Enables Iverson brackets
%----------------------------------------------------------------------------------------

\begin{document}

%-------------------------------
%	TITLE SECTION
%-------------------------------

\fancyhead[C]{}
\hrule \medskip % Upper rule
\begin{minipage}{0.295\textwidth} 
\raggedright
\footnotesize
Ryan Ott \hfill\\   
14862565 \hfill\\
ryan.ott@student.uva.nl
\end{minipage}
\begin{minipage}{0.4\textwidth} 
\centering 
\large 
Homework Assignment 4\\ 
\normalsize 
Machine Learning 1\\ 
\end{minipage}
\begin{minipage}{0.295\textwidth} 
\raggedleft
\today\hfill\\
\end{minipage}
\medskip\hrule 
\bigskip

%-------------------------------
%	CONTENTS
%-------------------------------
\section{Linear Module}
\subsection{} %a
\begin{align}
    \frac{\partial L}{\partial W} &= \sum_{i,j}^{} \frac{\partial L}{\partial Y_{ij}} \frac{\partial Y_{ij}}{\partial W_{mn}} \quad\quad \text{From section 3} \\
    \frac{\partial Y_{ij}}{\partial W_{mn}} &= \sum_{k} \frac{\partial}{\partial W_{mn}} (X_{ik} W_{kj}) = \sum_{k} X_{ik} \delta_{km} \delta_{jn} \\
    \frac{\partial L}{\partial W} = \sum_{i,j} \frac{\partial L}{\partial Y_{ij}} \frac{\partial Y_{ij}}{\partial W_{mn}} = \sum_{i,j,k} \frac{\partial L}{\partial Y_{ij}} X_{ik} \delta_{km} \delta_{jn} \\
    
\end{align}


3. Substituting this into the expression for \( \frac{\partial L}{\partial W} \):
   \[ \frac{\partial L}{\partial W} = \sum_{i,j} \frac{\partial L}{\partial Y_{ij}} \frac{\partial Y_{ij}}{\partial W_{mn}} = \sum_{i,j,k} \frac{\partial L}{\partial Y_{ij}} X_{ik} \delta_{km} \delta_{jn} \]

4. As the Kronecker deltas effectively filter for \( k = m \) and \( j = n \), the summation over \( k \) and \( j \) selects the \( m \)-th row and \( n \)-th column of \( X \) and \( \frac{\partial L}{\partial Y} \) respectively. Thus:
   \[ \frac{\partial L}{\partial W_{mn}} = \sum_{i} X_{im} \frac{\partial L}{\partial Y_{in}} \]

5. Rewriting in matrix terms, we arrive at the final expression:
   \[ \frac{\partial L}{\partial W} = \left( \frac{\partial L}{\partial Y} \right)^T X \]

This correctly captures the gradients of the loss with respect to the weights \( W \) in a linear module, using matrix calculus and the chain rule.
\end{document}