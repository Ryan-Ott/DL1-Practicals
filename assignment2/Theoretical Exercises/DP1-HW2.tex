\documentclass[a4paper]{article} 
\addtolength{\hoffset}{-2.25cm}
\addtolength{\textwidth}{4.5cm}
\addtolength{\voffset}{-3.25cm}
\addtolength{\textheight}{5cm}
\setlength{\parskip}{0pt}
\setlength{\parindent}{0in}

%----------------------------------------------------------------------------------------
%	PACKAGES AND OTHER DOCUMENT CONFIGURATIONS
%----------------------------------------------------------------------------------------
\usepackage{bm} % Bold math symbols, !must be loaded before unicode-math!
\usepackage{blindtext} % Package to generate dummy text
\usepackage{enumerate} % Enumerate with redefinable labels
\usepackage{enumitem} % Enumerate with redefinable labels
\renewcommand{\thesubsection}{\thesection.\alph{subsection}} % Enumerate subsections with letters
\renewcommand{\thesubsubsection}{\thesubsection.\roman{subsubsection}} % Enumerate subsubsections with roman numerals
\usepackage{charter} % Use the Charter font
\usepackage[utf8]{inputenc} % Use UTF-8 encoding
\usepackage{microtype} % Slightly tweak font spacing for aesthetics
\usepackage[english, ngerman]{babel} % Language hyphenation and typographical rules
\usepackage{amsthm, amsmath, amssymb} % Mathematical typesetting
\newcommand{\argmax}[1]{\underset{#1}{\text{arg max}}\;}
\newcommand{\argmin}[1]{\underset{#1}{\text{arg min}}\;}
\newcommand{\mb}[1]{\mathbf{#1}}  % Bold math symbols shorthand
\usepackage{float} % Improved interface for floating objects
\usepackage[final, colorlinks = true, 
            linkcolor = black, 
            citecolor = black]{hyperref} % For hyperlinks in the PDF
\usepackage{graphicx, multicol} % Enhanced support for graphics
\usepackage{xcolor} % Driver-independent color extensions
\usepackage{marvosym, wasysym} % More symbols
\usepackage{rotating} % Rotation tools
\usepackage{censor} % Facilities for controlling restricted text
\usepackage{listings, style/lstlisting} % Environment for non-formatted code, !uses style file!
\usepackage{pseudocode} % Environment for specifying algorithms in a natural way
\usepackage{style/avm} % Environment for f-structures, !uses style file!
\usepackage{booktabs} % Enhances quality of tables
\usepackage{tikz-qtree} % Easy tree drawing tool
\usepackage{todonotes} % Tool to insert TODOs
\tikzset{every tree node/.style={align=center,anchor=north},
         level distance=2cm} % Configuration for q-trees
\usepackage{style/btree} % Configuration for b-trees and b+-trees, !uses style file!
\usepackage[backend=biber,style=numeric,
            sorting=nyt]{biblatex} % Complete reimplementation of bibliographic facilities
% \addbibresource{ecl.bib}
\usepackage{csquotes} % Context sensitive quotation facilities
\usepackage[yyyymmdd]{datetime} % Uses YEAR-MONTH-DAY format for dates
\renewcommand{\dateseparator}{-} % Sets dateseparator to '-'
\usepackage{fancyhdr} % Headers and footers
\pagestyle{fancy} % All pages have headers and footers
\fancyhead{}\renewcommand{\headrulewidth}{0pt} % Blank out the default header
\fancyfoot[L]{} % Custom footer text
\fancyfoot[C]{} % Custom footer text
\fancyfoot[R]{\thepage} % Custom footer text
\newcommand{\note}[1]{\marginpar{\scriptsize \textcolor{red}{#1}}} % Enables comments in red on margin
\newcommand{\iverson}[1]{\ensuremath{[#1]}} % Enables Iverson brackets
%----------------------------------------------------------------------------------------

\begin{document}

%-------------------------------
%	TITLE SECTION
%-------------------------------

\fancyhead[C]{}
\hrule \medskip % Upper rule
\begin{minipage}{0.295\textwidth} 
\raggedright
\footnotesize
Ryan Ott \hfill\\   
14862565 \hfill\\
ryan.ott@student.uva.nl
\end{minipage}
\begin{minipage}{0.4\textwidth} 
\centering 
\large 
Practical Assignment 2\\ 
\normalsize 
Deep Learning 1\\ 
\end{minipage}
\begin{minipage}{0.295\textwidth} 
\raggedleft
\today\hfill\\
\end{minipage}
\medskip\hrule 
\bigskip

%-------------------------------
%	CONTENTS
%-------------------------------

\section{Transfer Learning}
\subsection{Comparing Models}
\subsubsection{Inference Speed per Image against Accuracy and Number of Parameters}
With a Pearson correlation coefficient of 0.85 there seems to be a strong positive linear correlation between the
top-1 accuracy and the inference speed. The ViT-B/32 model performs best overall with a higher accuracy for its
inference speed than the trend suggests.
For inference speed against number of trainable parameters, we see no correlation. The number of trainable parameters
should only play a role when the model is being trained, not during inferencing, so this result matches our
expectations.

\subsubsection{Inference Speed per Image with and without torch.no\_grad()}
With torch.no\_grad() the inference speed should be faster because gradients are not computed and stored for the
backward pass within the context manager's scope, thus requiring less operations and saving compute. This is
confirmed by the results, the inference speed is slightly faster with torch.no\_grad() across the models.

\subsubsection{vRAM Usage with and without torch.no\_grad()}
Like with the inference speed, the vRAM usage should be lower with torch.no\_grad() because gradients are not
stored for the backward pass when performing tensor operations, freeing up memory. Much more considerably than the
inference speed, the vRAM usage is lower when using torch.no\_grad() as seen in the plot.
\bigskip
% ------------------------------

\subsection{Fine-tuning}
\subsubsection{Retraining ResNet-18's Fully Connected Layer}
Using the specifications and default hyperparameters from the assignment, the retrained model achieved a test
accuracy of 0.5855 in CIFAR-100.

\subsubsection{Increasing Model Performance using Data Augmentation}
By applying \texttt{torchvision.transforms.RandomHorizontalFlip(p=0.5)} around half of the images in the training
set are flipped horizontally. This should increase the model's performance because it is being trained on more
varied data, thus making it more robust to different inputs. The model achieved a test accuracy of 0.5927, a small
improvement.

\subsubsection{Last vs First Convolutional Layer}
The first convolutional layers tend to capture lower-level features like edges, corners or textures, while the last
layers capture higher-level features like shapes and objects in a larger receptive field. As such, better performance
should be achievable by fine-tuning the last layers (along with the classifier layers) because they are more
specialized to the downstream task than the first layers.
\newpage
% ==============================

\section{Visual Prompting}
\subsection{CLIP Baseline}
\subsubsection{Top-1 Accuracy on CIFAR-10 and CIFAR-100 with CLIP ViT-B/32 Backbone}
% Zero-shot CLIP top-1 accuracy on cifar10/train: 88.726
% Zero-shot CLIP top-1 accuracy on cifar10/test: 88.94
% Zero-shot CLIP top-1 accuracy on cifar100/train: 63.564
% Zero-shot CLIP top-1 accuracy on cifar100/test: 63.14999999999999
\begin{table}[h]
    \centering
    \begin{tabular}{|l|c|c|}
    \hline
    \textbf{Dataset} & \textbf{Train Accuracy (\%)} & \textbf{Test Accuracy (\%)} \\ \hline
    CIFAR-10 & 88.726 & 88.940 \\ \hline
    CIFAR-100 & 63.564 & 63.150 \\ \hline
    \end{tabular}
    \caption{Zero-shot CLIP Top-1 Accuracy on CIFAR-10 and CIFAR-100}
    \label{tab:clip_accuracy}
\end{table}

\subsubsection{Prompting CLIP for a New Classification Task}
\textbf{Prompt:} \textit{"This is an image mostly coloured \_\_\_"}
\textbf{Labels:} \textit{"red", "green", "blue"}
--------------------------------
\textbf{Prompt:} \textit{"This image displays an object that is \_\_\_"}
\textbf{Labels:} \textit{"human-made", "from-nature"}

\end{document}